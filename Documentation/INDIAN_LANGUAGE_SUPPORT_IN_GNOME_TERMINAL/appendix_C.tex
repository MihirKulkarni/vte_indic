\chapter{Glossary}
\begin{enumerate}
\item Diacritics

A diacritic is an ancillary glyph added to a letter or basic glyph. Diacritical marks may appear above or below a letter, or in some other position such as within the letter or between two letters.

\begin{figure}[htbp]
%place the figure according to your convenience
\centerline{\epsfig{figure=diacritics.eps}}
\caption{Example of diacritics} \label{Gnome Terminal15}
\end{figure}

\item Glyph

A glyph is a graphical representation of an element of writing that contributes to meaning of what is written. A single character may have different glyph in different font. Below figure shows glyphs for Latin character ``a'' in different fonts.

\begin{figure}[htbp]
%place the figure according to your convenience
\centerline{\epsfig{figure=glyphs.eps}}
\caption{Example of glyph} \label{Gnome Terminal19}
\end{figure}

\item Font

A font a contains a set of images representing the glyphs, characters of a script and other information such as scaling etc.

\item Zero-width character

A zero-width character is one which when attached to the base glyph does not require additional width to draw.

\begin{figure}[htbp]
%place the figure according to your convenience
\centerline{\epsfig{figure=zero_width.eps}}
\caption{Example of zero-width character} \label{Gnome Terminal16}
\end{figure}
  
%\pagebreak
\item Non-zero-width character

A Non-zero-width character is one which when attached to the base glyph requires additional width to draw. 

\begin{figure}[htbp]
%place the figure according to your convenience
\centerline{\epsfig{figure=non_zero_width.eps}}
\caption{Example of non-zero-width character} \label{Gnome Terminal17}
\end{figure}

\item Halant

A \textit{halant} character is one when used between two characters joins them and forms a combined character

\item Conjunct

A conjunct is a combined character formed by joining two base consonants characters using a \textit{halant} character.

\begin{figure}[htbp]
%place the figure according to your convenience
\centerline{\epsfig{figure=conjunct.eps}}
\caption{Example of conjunct} \label{Gnome Terminal18}
\end{figure}

\item GNOME

GNOME is a free software project consisting of The GNOME desktop environment, a attractive desktop for users and the GNOME development platform, an extensive framework for building applications that integrate into the the desktop. It is the most popular desktop environment for GNU\slash Linux and UNIX-type operating systems.

\item KDE

KDE is a free software project that offers an advanced graphical desktop, a wide variety of cross-platform applications and a platform to easily build new applications upon. KDE software is based on the Qt framework.

\item Qt

Qt is a non-GPL, cross-platform application framework that is used for developing application software with a graphical user interface (GUI) hence it is also called as Qt widget toolkit. It is also used for developing non-GUI tools such as command-line tools and consoles for servers.

\item Pango

Pango is an LGPL licensed library for laying out and rendering of text, with an emphasis on internationalization. The GTK+ UI toolkit uses pango for all of its text rendering. It provides cross-platform support, so that pango-rendered text will appear similar under different operating systems, such as LINUX, Apple's Mac OS and Microsoft Windows.

\item GTK- GIMP Tool Kit

GTK is a LGPL licensed widget toolkit for creating graphical user interfaces and provides cross platform compatibility and an easy to use API.

\end{enumerate}