%_____________________________________________________________________________________________ 
% LATEX Template: Department of Comp/IT BTech Project Reports
% Sample Chapter
% Sun Mar 27 10:25:35 IST 2011
%
% Note: Itemization, enumeration and other things not shown. A sample figure is included.
%_____________________________________________________________________________________________ 

\chapter{Introduction}
\section{Overview}
%This is a section. We can cite a reference like this: \cite{INTERNET} 	
						% Citation. See references.tex for the entry.

Computer is becoming the integral part of the human life. Almost all the industrial processes are using the computers in direct or indirect manner. Computers are used in our education system, government offices, etc. However most of the computing systems are developed with English as default language. Though many European and some Asian languages are increasingly supported, the support for Indic languages remains quite low. Common Indian people need local language support for becoming computer literate.

Any user- novice or expert, who uses GNU/Linux with GNOME desktop environment knows the importance of \textit{vim-editor} and \textit{GNOME-terminal}. These are the most common  applications used in GNU/Linux. These applications are having complex script rendering problem since they were written. There is a demand from all over the world for complex script support in these applications.

In the recent years there has been lot of development going on for making of GNU\slash Linux software available in local languages. For languages like Malayalam there is much work done and going on too. There is a large need for making similar support available for Marathi/Hindi also.


Indic Languages is a term which is commonly referred to languages that are spoken in India. These include languages like Hindi, Marathi, Gujarati, Malayalam, Kannada, Tamil, etc. Indic Language Computing thus refers to all the work which is being done worldwide to develop new software and to increase support for Indic Languages on existing software.


Moving ahead with this, Devanagari Script \cite{wiki1} is the script which is used by Devanagari languages like Hindi, Marathi and Sanskrit. Devanagari Script is also referred to as a complex script \cite{wiki2} because of the nature of its writing. Unlike English where there in no change in the characters during formation of words, in complex scripts there are changes that have to be done to the previous character itself to produce change of sound/syllable. Some examples of complex scripts are languages like Hindi, Marathi, etc. A problem that is being faced is that of rendering of this Devanagari Script on the \textit{GNOME-terminal}. We have taken this problem as our task.

Improving and providing Devanagari Script support on \textit{GNOME-terminal} will make it very useful for people who are familiar with the Devanagari Script but are not comfortable with English. Also, we are hopeful that our efforts will help other developers who are trying to develop support in the \textit{GNOME-terminal} for their own local languages.

\section{Objective}
The main objective of our project was to develop support for Devanagari Script in the \textit{GNOME-terminal}. This involved improving the rendering of Devanagari Script which at present is not proper.

%\begin{figure}[htbp]			 Sample figure 
%\begin{center}
%\input{fig1.latex}			 Be sure to have the input file in the directory
%\caption{A simple figure: Square}	 This will appear in the list of figures
%\label{circle}
%\end{center}
%\end{figure}

%_____________________________________________________________________________________________ 
